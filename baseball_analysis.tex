% Options for packages loaded elsewhere
\PassOptionsToPackage{unicode}{hyperref}
\PassOptionsToPackage{hyphens}{url}
%
\documentclass[
]{article}
\usepackage{lmodern}
\usepackage{amssymb,amsmath}
\usepackage{ifxetex,ifluatex}
\ifnum 0\ifxetex 1\fi\ifluatex 1\fi=0 % if pdftex
  \usepackage[T1]{fontenc}
  \usepackage[utf8]{inputenc}
  \usepackage{textcomp} % provide euro and other symbols
\else % if luatex or xetex
  \usepackage{unicode-math}
  \defaultfontfeatures{Scale=MatchLowercase}
  \defaultfontfeatures[\rmfamily]{Ligatures=TeX,Scale=1}
\fi
% Use upquote if available, for straight quotes in verbatim environments
\IfFileExists{upquote.sty}{\usepackage{upquote}}{}
\IfFileExists{microtype.sty}{% use microtype if available
  \usepackage[]{microtype}
  \UseMicrotypeSet[protrusion]{basicmath} % disable protrusion for tt fonts
}{}
\makeatletter
\@ifundefined{KOMAClassName}{% if non-KOMA class
  \IfFileExists{parskip.sty}{%
    \usepackage{parskip}
  }{% else
    \setlength{\parindent}{0pt}
    \setlength{\parskip}{6pt plus 2pt minus 1pt}}
}{% if KOMA class
  \KOMAoptions{parskip=half}}
\makeatother
\usepackage{xcolor}
\IfFileExists{xurl.sty}{\usepackage{xurl}}{} % add URL line breaks if available
\IfFileExists{bookmark.sty}{\usepackage{bookmark}}{\usepackage{hyperref}}
\hypersetup{
  pdftitle={Exploring R},
  pdfauthor={Shawn Santo},
  hidelinks,
  pdfcreator={LaTeX via pandoc}}
\urlstyle{same} % disable monospaced font for URLs
\usepackage[margin=1in]{geometry}
\usepackage{color}
\usepackage{fancyvrb}
\newcommand{\VerbBar}{|}
\newcommand{\VERB}{\Verb[commandchars=\\\{\}]}
\DefineVerbatimEnvironment{Highlighting}{Verbatim}{commandchars=\\\{\}}
% Add ',fontsize=\small' for more characters per line
\usepackage{framed}
\definecolor{shadecolor}{RGB}{248,248,248}
\newenvironment{Shaded}{\begin{snugshade}}{\end{snugshade}}
\newcommand{\AlertTok}[1]{\textcolor[rgb]{0.94,0.16,0.16}{#1}}
\newcommand{\AnnotationTok}[1]{\textcolor[rgb]{0.56,0.35,0.01}{\textbf{\textit{#1}}}}
\newcommand{\AttributeTok}[1]{\textcolor[rgb]{0.77,0.63,0.00}{#1}}
\newcommand{\BaseNTok}[1]{\textcolor[rgb]{0.00,0.00,0.81}{#1}}
\newcommand{\BuiltInTok}[1]{#1}
\newcommand{\CharTok}[1]{\textcolor[rgb]{0.31,0.60,0.02}{#1}}
\newcommand{\CommentTok}[1]{\textcolor[rgb]{0.56,0.35,0.01}{\textit{#1}}}
\newcommand{\CommentVarTok}[1]{\textcolor[rgb]{0.56,0.35,0.01}{\textbf{\textit{#1}}}}
\newcommand{\ConstantTok}[1]{\textcolor[rgb]{0.00,0.00,0.00}{#1}}
\newcommand{\ControlFlowTok}[1]{\textcolor[rgb]{0.13,0.29,0.53}{\textbf{#1}}}
\newcommand{\DataTypeTok}[1]{\textcolor[rgb]{0.13,0.29,0.53}{#1}}
\newcommand{\DecValTok}[1]{\textcolor[rgb]{0.00,0.00,0.81}{#1}}
\newcommand{\DocumentationTok}[1]{\textcolor[rgb]{0.56,0.35,0.01}{\textbf{\textit{#1}}}}
\newcommand{\ErrorTok}[1]{\textcolor[rgb]{0.64,0.00,0.00}{\textbf{#1}}}
\newcommand{\ExtensionTok}[1]{#1}
\newcommand{\FloatTok}[1]{\textcolor[rgb]{0.00,0.00,0.81}{#1}}
\newcommand{\FunctionTok}[1]{\textcolor[rgb]{0.00,0.00,0.00}{#1}}
\newcommand{\ImportTok}[1]{#1}
\newcommand{\InformationTok}[1]{\textcolor[rgb]{0.56,0.35,0.01}{\textbf{\textit{#1}}}}
\newcommand{\KeywordTok}[1]{\textcolor[rgb]{0.13,0.29,0.53}{\textbf{#1}}}
\newcommand{\NormalTok}[1]{#1}
\newcommand{\OperatorTok}[1]{\textcolor[rgb]{0.81,0.36,0.00}{\textbf{#1}}}
\newcommand{\OtherTok}[1]{\textcolor[rgb]{0.56,0.35,0.01}{#1}}
\newcommand{\PreprocessorTok}[1]{\textcolor[rgb]{0.56,0.35,0.01}{\textit{#1}}}
\newcommand{\RegionMarkerTok}[1]{#1}
\newcommand{\SpecialCharTok}[1]{\textcolor[rgb]{0.00,0.00,0.00}{#1}}
\newcommand{\SpecialStringTok}[1]{\textcolor[rgb]{0.31,0.60,0.02}{#1}}
\newcommand{\StringTok}[1]{\textcolor[rgb]{0.31,0.60,0.02}{#1}}
\newcommand{\VariableTok}[1]{\textcolor[rgb]{0.00,0.00,0.00}{#1}}
\newcommand{\VerbatimStringTok}[1]{\textcolor[rgb]{0.31,0.60,0.02}{#1}}
\newcommand{\WarningTok}[1]{\textcolor[rgb]{0.56,0.35,0.01}{\textbf{\textit{#1}}}}
\usepackage{graphicx,grffile}
\makeatletter
\def\maxwidth{\ifdim\Gin@nat@width>\linewidth\linewidth\else\Gin@nat@width\fi}
\def\maxheight{\ifdim\Gin@nat@height>\textheight\textheight\else\Gin@nat@height\fi}
\makeatother
% Scale images if necessary, so that they will not overflow the page
% margins by default, and it is still possible to overwrite the defaults
% using explicit options in \includegraphics[width, height, ...]{}
\setkeys{Gin}{width=\maxwidth,height=\maxheight,keepaspectratio}
% Set default figure placement to htbp
\makeatletter
\def\fps@figure{htbp}
\makeatother
\setlength{\emergencystretch}{3em} % prevent overfull lines
\providecommand{\tightlist}{%
  \setlength{\itemsep}{0pt}\setlength{\parskip}{0pt}}
\setcounter{secnumdepth}{-\maxdimen} % remove section numbering

\title{Exploring R}
\author{Shawn Santo}
\date{4/14/2021}

\begin{document}
\maketitle

\hypertarget{packages}{%
\subsection{Packages}\label{packages}}

\begin{Shaded}
\begin{Highlighting}[]
\KeywordTok{library}\NormalTok{(tidyverse)}
\KeywordTok{library}\NormalTok{(Lahman)}
\KeywordTok{library}\NormalTok{(broom)}
\end{Highlighting}
\end{Shaded}

\hypertarget{introduction-a-quick-primer-on-rstudio-hooray}{%
\subsection{Introduction (a quick primer on RStudio
hooray)}\label{introduction-a-quick-primer-on-rstudio-hooray}}

\hypertarget{vectors}{%
\subsubsection{Vectors}\label{vectors}}

A vector is the basic building block in R.

\begin{Shaded}
\begin{Highlighting}[]
\KeywordTok{c}\NormalTok{(}\DecValTok{4}\NormalTok{, }\DecValTok{1}\NormalTok{, }\DecValTok{0}\NormalTok{)}
\end{Highlighting}
\end{Shaded}

\begin{verbatim}
## [1] 4 1 0
\end{verbatim}

\begin{Shaded}
\begin{Highlighting}[]
\KeywordTok{c}\NormalTok{(}\StringTok{"intro"}\NormalTok{, }\StringTok{"to"}\NormalTok{, }\StringTok{"r"}\NormalTok{)}
\end{Highlighting}
\end{Shaded}

\begin{verbatim}
## [1] "intro" "to"    "r"
\end{verbatim}

\begin{Shaded}
\begin{Highlighting}[]
\KeywordTok{c}\NormalTok{(pi, }\DecValTok{5} \OperatorTok{/}\StringTok{ }\DecValTok{2}\NormalTok{, }\DecValTok{6} \OperatorTok{*}\StringTok{ }\DecValTok{4}\NormalTok{, }\DecValTok{-3}\NormalTok{)}
\end{Highlighting}
\end{Shaded}

\begin{verbatim}
## [1]  3.141593  2.500000 24.000000 -3.000000
\end{verbatim}

The four main vector types you will work with are \texttt{logical},
\texttt{integer}, \texttt{double}, and \texttt{character}.

Some vector properties:

\begin{enumerate}
\def\labelenumi{\arabic{enumi}.}
\item
  Each element of a vector must be the same type, check the type with
  \texttt{typeof()}
\item
  Vector indexing begins at 1, not 0
\item
  Vectors can be subset by position
\end{enumerate}

\hypertarget{data-frames-tibbles}{%
\subsubsection{Data frames (tibbles)}\label{data-frames-tibbles}}

A data frame (or tibble) is a rectangular structure in R where each
column is a vector. It is a very common structure to work with in doing
data wrangling and visualization.

\begin{Shaded}
\begin{Highlighting}[]
\NormalTok{mtcars}
\end{Highlighting}
\end{Shaded}

\begin{verbatim}
##                      mpg cyl  disp  hp drat    wt  qsec vs am gear carb
## Mazda RX4           21.0   6 160.0 110 3.90 2.620 16.46  0  1    4    4
## Mazda RX4 Wag       21.0   6 160.0 110 3.90 2.875 17.02  0  1    4    4
## Datsun 710          22.8   4 108.0  93 3.85 2.320 18.61  1  1    4    1
## Hornet 4 Drive      21.4   6 258.0 110 3.08 3.215 19.44  1  0    3    1
## Hornet Sportabout   18.7   8 360.0 175 3.15 3.440 17.02  0  0    3    2
## Valiant             18.1   6 225.0 105 2.76 3.460 20.22  1  0    3    1
## Duster 360          14.3   8 360.0 245 3.21 3.570 15.84  0  0    3    4
## Merc 240D           24.4   4 146.7  62 3.69 3.190 20.00  1  0    4    2
## Merc 230            22.8   4 140.8  95 3.92 3.150 22.90  1  0    4    2
## Merc 280            19.2   6 167.6 123 3.92 3.440 18.30  1  0    4    4
## Merc 280C           17.8   6 167.6 123 3.92 3.440 18.90  1  0    4    4
## Merc 450SE          16.4   8 275.8 180 3.07 4.070 17.40  0  0    3    3
## Merc 450SL          17.3   8 275.8 180 3.07 3.730 17.60  0  0    3    3
## Merc 450SLC         15.2   8 275.8 180 3.07 3.780 18.00  0  0    3    3
## Cadillac Fleetwood  10.4   8 472.0 205 2.93 5.250 17.98  0  0    3    4
## Lincoln Continental 10.4   8 460.0 215 3.00 5.424 17.82  0  0    3    4
## Chrysler Imperial   14.7   8 440.0 230 3.23 5.345 17.42  0  0    3    4
## Fiat 128            32.4   4  78.7  66 4.08 2.200 19.47  1  1    4    1
## Honda Civic         30.4   4  75.7  52 4.93 1.615 18.52  1  1    4    2
## Toyota Corolla      33.9   4  71.1  65 4.22 1.835 19.90  1  1    4    1
## Toyota Corona       21.5   4 120.1  97 3.70 2.465 20.01  1  0    3    1
## Dodge Challenger    15.5   8 318.0 150 2.76 3.520 16.87  0  0    3    2
## AMC Javelin         15.2   8 304.0 150 3.15 3.435 17.30  0  0    3    2
## Camaro Z28          13.3   8 350.0 245 3.73 3.840 15.41  0  0    3    4
## Pontiac Firebird    19.2   8 400.0 175 3.08 3.845 17.05  0  0    3    2
## Fiat X1-9           27.3   4  79.0  66 4.08 1.935 18.90  1  1    4    1
## Porsche 914-2       26.0   4 120.3  91 4.43 2.140 16.70  0  1    5    2
## Lotus Europa        30.4   4  95.1 113 3.77 1.513 16.90  1  1    5    2
## Ford Pantera L      15.8   8 351.0 264 4.22 3.170 14.50  0  1    5    4
## Ferrari Dino        19.7   6 145.0 175 3.62 2.770 15.50  0  1    5    6
## Maserati Bora       15.0   8 301.0 335 3.54 3.570 14.60  0  1    5    8
## Volvo 142E          21.4   4 121.0 109 4.11 2.780 18.60  1  1    4    2
\end{verbatim}

\begin{Shaded}
\begin{Highlighting}[]
\NormalTok{storms}
\end{Highlighting}
\end{Shaded}

\begin{verbatim}
## # A tibble: 10,010 x 13
##    name   year month   day  hour   lat  long status      category  wind pressure
##    <chr> <dbl> <dbl> <int> <dbl> <dbl> <dbl> <chr>       <ord>    <int>    <int>
##  1 Amy    1975     6    27     0  27.5 -79   tropical d~ -1          25     1013
##  2 Amy    1975     6    27     6  28.5 -79   tropical d~ -1          25     1013
##  3 Amy    1975     6    27    12  29.5 -79   tropical d~ -1          25     1013
##  4 Amy    1975     6    27    18  30.5 -79   tropical d~ -1          25     1013
##  5 Amy    1975     6    28     0  31.5 -78.8 tropical d~ -1          25     1012
##  6 Amy    1975     6    28     6  32.4 -78.7 tropical d~ -1          25     1012
##  7 Amy    1975     6    28    12  33.3 -78   tropical d~ -1          25     1011
##  8 Amy    1975     6    28    18  34   -77   tropical d~ -1          30     1006
##  9 Amy    1975     6    29     0  34.4 -75.8 tropical s~ 0           35     1004
## 10 Amy    1975     6    29     6  34   -74.8 tropical s~ 0           40     1002
## # ... with 10,000 more rows, and 2 more variables: ts_diameter <dbl>,
## #   hu_diameter <dbl>
\end{verbatim}

A tibble is a nicer form of a data frame, so generally we will coerce
data frames to tibbles with \texttt{as\_tibble()}.

\hypertarget{object-creation}{%
\subsubsection{Object creation}\label{object-creation}}

Objects can be created with \texttt{\textless{}-} or \texttt{=}.
However, using the assignment operator, \texttt{\textless{}-} is
considered best practice.

\begin{Shaded}
\begin{Highlighting}[]
\NormalTok{a <-}\StringTok{ }\DecValTok{10} \OperatorTok{^}\StringTok{ }\DecValTok{4}
\NormalTok{b <-}\StringTok{ }\KeywordTok{sqrt}\NormalTok{(pi)}

\NormalTok{total <-}\StringTok{ }\KeywordTok{sum}\NormalTok{(}\KeywordTok{c}\NormalTok{(}\DecValTok{4}\NormalTok{, }\DecValTok{5}\NormalTok{, }\DecValTok{0}\NormalTok{, }\DecValTok{10}\NormalTok{))}
\NormalTok{coin_tosses <-}\StringTok{ }\KeywordTok{sample}\NormalTok{(}\KeywordTok{c}\NormalTok{(}\StringTok{"H"}\NormalTok{, }\StringTok{"T"}\NormalTok{), }\DataTypeTok{size =} \DecValTok{12}\NormalTok{, }\DataTypeTok{replace =} \OtherTok{TRUE}\NormalTok{)}
\end{Highlighting}
\end{Shaded}

\begin{Shaded}
\begin{Highlighting}[]
\NormalTok{a}
\end{Highlighting}
\end{Shaded}

\begin{verbatim}
## [1] 10000
\end{verbatim}

\begin{Shaded}
\begin{Highlighting}[]
\NormalTok{b}
\end{Highlighting}
\end{Shaded}

\begin{verbatim}
## [1] 1.772454
\end{verbatim}

\begin{Shaded}
\begin{Highlighting}[]
\NormalTok{total}
\end{Highlighting}
\end{Shaded}

\begin{verbatim}
## [1] 19
\end{verbatim}

\begin{Shaded}
\begin{Highlighting}[]
\NormalTok{coin_tosses}
\end{Highlighting}
\end{Shaded}

\begin{verbatim}
##  [1] "T" "T" "H" "T" "H" "T" "T" "H" "T" "H" "H" "H"
\end{verbatim}

\hypertarget{data-wrangling}{%
\subsection{Data wrangling}\label{data-wrangling}}

The \texttt{dplyr} package is a package in \texttt{tidyverse} that is
great for data wrangling data frame objects.

We'll work with some baseball data from the \texttt{Lahmen} package.

\begin{Shaded}
\begin{Highlighting}[]
\NormalTok{teams <-}\StringTok{ }\KeywordTok{as_tibble}\NormalTok{(Teams)}
\end{Highlighting}
\end{Shaded}

Take a \texttt{glimpse()} at \texttt{teams}.

\begin{Shaded}
\begin{Highlighting}[]
\KeywordTok{glimpse}\NormalTok{(teams)}
\end{Highlighting}
\end{Shaded}

\begin{verbatim}
## Rows: 2,955
## Columns: 48
## $ yearID         <int> 1871, 1871, 1871, 1871, 1871, 1871, 1871, 1871, 1871, 1~
## $ lgID           <fct> NA, NA, NA, NA, NA, NA, NA, NA, NA, NA, NA, NA, NA, NA,~
## $ teamID         <fct> BS1, CH1, CL1, FW1, NY2, PH1, RC1, TRO, WS3, BL1, BR1, ~
## $ franchID       <fct> BNA, CNA, CFC, KEK, NNA, PNA, ROK, TRO, OLY, BLC, ECK, ~
## $ divID          <chr> NA, NA, NA, NA, NA, NA, NA, NA, NA, NA, NA, NA, NA, NA,~
## $ Rank           <int> 3, 2, 8, 7, 5, 1, 9, 6, 4, 2, 9, 6, 1, 7, 8, 3, 4, 5, 1~
## $ G              <int> 31, 28, 29, 19, 33, 28, 25, 29, 32, 58, 29, 37, 48, 22,~
## $ Ghome          <int> NA, NA, NA, NA, NA, NA, NA, NA, NA, NA, NA, NA, NA, NA,~
## $ W              <int> 20, 19, 10, 7, 16, 21, 4, 13, 15, 35, 3, 9, 39, 6, 5, 3~
## $ L              <int> 10, 9, 19, 12, 17, 7, 21, 15, 15, 19, 26, 28, 8, 16, 19~
## $ DivWin         <chr> NA, NA, NA, NA, NA, NA, NA, NA, NA, NA, NA, NA, NA, NA,~
## $ WCWin          <chr> NA, NA, NA, NA, NA, NA, NA, NA, NA, NA, NA, NA, NA, NA,~
## $ LgWin          <chr> "N", "N", "N", "N", "N", "Y", "N", "N", "N", "N", "N", ~
## $ WSWin          <chr> NA, NA, NA, NA, NA, NA, NA, NA, NA, NA, NA, NA, NA, NA,~
## $ R              <int> 401, 302, 249, 137, 302, 376, 231, 351, 310, 617, 152, ~
## $ AB             <int> 1372, 1196, 1186, 746, 1404, 1281, 1036, 1248, 1353, 25~
## $ H              <int> 426, 323, 328, 178, 403, 410, 274, 384, 375, 753, 248, ~
## $ X2B            <int> 70, 52, 35, 19, 43, 66, 44, 51, 54, 106, 29, 35, 107, 2~
## $ X3B            <int> 37, 21, 40, 8, 21, 27, 25, 34, 26, 31, 9, 10, 30, 5, 9,~
## $ HR             <int> 3, 10, 7, 2, 1, 9, 3, 6, 6, 14, 0, 1, 7, 0, 2, 4, 4, 5,~
## $ BB             <int> 60, 60, 26, 33, 33, 46, 38, 49, 48, 29, 18, 19, 29, 17,~
## $ SO             <int> 19, 22, 25, 9, 15, 23, 30, 19, 13, 28, 40, 25, 26, 13, ~
## $ SB             <int> 73, 69, 18, 16, 46, 56, 53, 62, 48, 53, 8, 19, 48, 12, ~
## $ CS             <int> 16, 21, 8, 4, 15, 12, 10, 24, 13, 18, 13, 16, 14, 3, 7,~
## $ HBP            <int> NA, NA, NA, NA, NA, NA, NA, NA, NA, NA, NA, NA, NA, NA,~
## $ SF             <int> NA, NA, NA, NA, NA, NA, NA, NA, NA, NA, NA, NA, NA, NA,~
## $ RA             <int> 303, 241, 341, 243, 313, 266, 287, 362, 303, 434, 413, ~
## $ ER             <int> 109, 77, 116, 97, 121, 137, 108, 153, 137, 166, 160, 16~
## $ ERA            <dbl> 3.55, 2.76, 4.11, 5.17, 3.72, 4.95, 4.30, 5.51, 4.37, 2~
## $ CG             <int> 22, 25, 23, 19, 32, 27, 23, 28, 32, 48, 28, 37, 41, 15,~
## $ SHO            <int> 1, 0, 0, 1, 1, 0, 1, 0, 0, 1, 0, 0, 4, 0, 0, 3, 1, 2, 0~
## $ SV             <int> 3, 1, 0, 0, 0, 0, 0, 0, 0, 1, 0, 0, 4, 0, 0, 1, 0, 1, 0~
## $ IPouts         <int> 828, 753, 762, 507, 879, 747, 678, 750, 846, 1548, 778,~
## $ HA             <int> 367, 308, 346, 261, 373, 329, 315, 431, 371, 573, 484, ~
## $ HRA            <int> 2, 6, 13, 5, 7, 3, 3, 4, 4, 3, 7, 6, 0, 6, 6, 2, 3, 2, ~
## $ BBA            <int> 42, 28, 53, 21, 42, 53, 34, 75, 45, 63, 36, 21, 27, 24,~
## $ SOA            <int> 23, 22, 34, 17, 22, 16, 16, 12, 13, 77, 13, 13, 29, 11,~
## $ E              <int> 243, 229, 234, 163, 235, 194, 220, 198, 218, 432, 274, ~
## $ DP             <int> 24, 16, 15, 8, 14, 13, 14, 22, 20, 22, 9, 15, 44, 17, 1~
## $ FP             <dbl> 0.834, 0.829, 0.818, 0.803, 0.840, 0.845, 0.821, 0.845,~
## $ name           <chr> "Boston Red Stockings", "Chicago White Stockings", "Cle~
## $ park           <chr> "South End Grounds I", "Union Base-Ball Grounds", "Nati~
## $ attendance     <int> NA, NA, NA, NA, NA, NA, NA, NA, NA, NA, NA, NA, NA, NA,~
## $ BPF            <int> 103, 104, 96, 101, 90, 102, 97, 101, 94, 106, 87, 115, ~
## $ PPF            <int> 98, 102, 100, 107, 88, 98, 99, 100, 98, 102, 96, 122, 1~
## $ teamIDBR       <chr> "BOS", "CHI", "CLE", "KEK", "NYU", "ATH", "ROK", "TRO",~
## $ teamIDlahman45 <chr> "BS1", "CH1", "CL1", "FW1", "NY2", "PH1", "RC1", "TRO",~
## $ teamIDretro    <chr> "BS1", "CH1", "CL1", "FW1", "NY2", "PH1", "RC1", "TRO",~
\end{verbatim}

To do further manipulations we need to introduce the pipe operator -
\texttt{\%\textgreater{}\%}. This operator takes the object on the
left-hand-side and passes it into the object on the right-hand-side.
Rather than write \texttt{f(x,\ y)} one could write
\texttt{x\ \%\textgreater{}\%\ f(y)}.

\hypertarget{select}{%
\subsubsection{Select}\label{select}}

Let's \texttt{select()} the runs, runs allowed, wins, year, and team
name.

\begin{Shaded}
\begin{Highlighting}[]
\NormalTok{teams }\OperatorTok\StringTok{ }
\StringTok{  }\KeywordTok{select}\NormalTok{(R, RA, W, yearID, name)}
\end{Highlighting}
\end{Shaded}

\begin{verbatim}
## # A tibble: 2,955 x 5
##        R    RA     W yearID name                   
##    <int> <int> <int>  <int> <chr>                  
##  1   401   303    20   1871 Boston Red Stockings   
##  2   302   241    19   1871 Chicago White Stockings
##  3   249   341    10   1871 Cleveland Forest Citys 
##  4   137   243     7   1871 Fort Wayne Kekiongas   
##  5   302   313    16   1871 New York Mutuals       
##  6   376   266    21   1871 Philadelphia Athletics 
##  7   231   287     4   1871 Rockford Forest Citys  
##  8   351   362    13   1871 Troy Haymakers         
##  9   310   303    15   1871 Washington Olympics    
## 10   617   434    35   1872 Baltimore Canaries     
## # ... with 2,945 more rows
\end{verbatim}

\texttt{select()} allows you to choose variables (columns). Separate
your choices with a comma.

\hypertarget{filter}{%
\subsubsection{Filter}\label{filter}}

Let's \texttt{filter()} \texttt{teams} for those years since 1970.

\begin{Shaded}
\begin{Highlighting}[]
\NormalTok{teams }\OperatorTok\StringTok{ }
\StringTok{  }\KeywordTok{filter}\NormalTok{(yearID }\OperatorTok{>=}\StringTok{ }\DecValTok{1970}\NormalTok{)}
\end{Highlighting}
\end{Shaded}

\begin{verbatim}
## # A tibble: 1,414 x 48
##    yearID lgID  teamID franchID divID  Rank     G Ghome     W     L DivWin WCWin
##     <int> <fct> <fct>  <fct>    <chr> <int> <int> <int> <int> <int> <chr>  <chr>
##  1   1970 NL    ATL    ATL      W         5   162    81    76    86 N      <NA> 
##  2   1970 AL    BAL    BAL      E         1   162    81   108    54 Y      <NA> 
##  3   1970 AL    BOS    BOS      E         3   162    81    87    75 N      <NA> 
##  4   1970 AL    CAL    ANA      W         3   162    81    86    76 N      <NA> 
##  5   1970 AL    CHA    CHW      W         6   162    84    56   106 N      <NA> 
##  6   1970 NL    CHN    CHC      E         2   162    80    84    78 N      <NA> 
##  7   1970 NL    CIN    CIN      W         1   162    81   102    60 Y      <NA> 
##  8   1970 AL    CLE    CLE      E         5   162    81    76    86 N      <NA> 
##  9   1970 AL    DET    DET      E         4   162    81    79    83 N      <NA> 
## 10   1970 NL    HOU    HOU      W         4   162    81    79    83 N      <NA> 
## # ... with 1,404 more rows, and 36 more variables: LgWin <chr>, WSWin <chr>,
## #   R <int>, AB <int>, H <int>, X2B <int>, X3B <int>, HR <int>, BB <int>,
## #   SO <int>, SB <int>, CS <int>, HBP <int>, SF <int>, RA <int>, ER <int>,
## #   ERA <dbl>, CG <int>, SHO <int>, SV <int>, IPouts <int>, HA <int>,
## #   HRA <int>, BBA <int>, SOA <int>, E <int>, DP <int>, FP <dbl>, name <chr>,
## #   park <chr>, attendance <int>, BPF <int>, PPF <int>, teamIDBR <chr>,
## #   teamIDlahman45 <chr>, teamIDretro <chr>
\end{verbatim}

\texttt{filter()} chooses rows based on conditions of variables.

\hypertarget{arrange}{%
\subsubsection{Arrange}\label{arrange}}

Let's \texttt{arrange()} \texttt{teams} in descending order by wins.

\begin{Shaded}
\begin{Highlighting}[]
\NormalTok{teams }\OperatorTok\StringTok{ }
\StringTok{  }\KeywordTok{arrange}\NormalTok{(}\KeywordTok{desc}\NormalTok{(W))}
\end{Highlighting}
\end{Shaded}

\begin{verbatim}
## # A tibble: 2,955 x 48
##    yearID lgID  teamID franchID divID  Rank     G Ghome     W     L DivWin WCWin
##     <int> <fct> <fct>  <fct>    <chr> <int> <int> <int> <int> <int> <chr>  <chr>
##  1   1906 NL    CHN    CHC      <NA>      1   154    79   116    36 <NA>   <NA> 
##  2   2001 AL    SEA    SEA      W         1   162    81   116    46 Y      N    
##  3   1998 AL    NYA    NYY      E         1   162    81   114    48 Y      N    
##  4   1954 AL    CLE    CLE      <NA>      1   156    77   111    43 <NA>   <NA> 
##  5   1909 NL    PIT    PIT      <NA>      1   154    78   110    42 <NA>   <NA> 
##  6   1927 AL    NYA    NYY      <NA>      1   155    77   110    44 <NA>   <NA> 
##  7   1961 AL    NYA    NYY      <NA>      1   163    81   109    53 <NA>   <NA> 
##  8   1969 AL    BAL    BAL      E         1   162    81   109    53 Y      <NA> 
##  9   1970 AL    BAL    BAL      E         1   162    81   108    54 Y      <NA> 
## 10   1975 NL    CIN    CIN      W         1   162    81   108    54 Y      <NA> 
## # ... with 2,945 more rows, and 36 more variables: LgWin <chr>, WSWin <chr>,
## #   R <int>, AB <int>, H <int>, X2B <int>, X3B <int>, HR <int>, BB <int>,
## #   SO <int>, SB <int>, CS <int>, HBP <int>, SF <int>, RA <int>, ER <int>,
## #   ERA <dbl>, CG <int>, SHO <int>, SV <int>, IPouts <int>, HA <int>,
## #   HRA <int>, BBA <int>, SOA <int>, E <int>, DP <int>, FP <dbl>, name <chr>,
## #   park <chr>, attendance <int>, BPF <int>, PPF <int>, teamIDBR <chr>,
## #   teamIDlahman45 <chr>, teamIDretro <chr>
\end{verbatim}

The default sorting with \texttt{arrange()} is ascending.

\hypertarget{mutate}{%
\subsubsection{Mutate}\label{mutate}}

Let's create a new variable in \texttt{teams} called win percentage.

\begin{Shaded}
\begin{Highlighting}[]
\NormalTok{teams }\OperatorTok\StringTok{ }
\StringTok{  }\KeywordTok{mutate}\NormalTok{(}\DataTypeTok{win_pct =}\NormalTok{ W }\OperatorTok{/}\StringTok{ }\NormalTok{G)}
\end{Highlighting}
\end{Shaded}

\begin{verbatim}
## # A tibble: 2,955 x 49
##    yearID lgID  teamID franchID divID  Rank     G Ghome     W     L DivWin WCWin
##     <int> <fct> <fct>  <fct>    <chr> <int> <int> <int> <int> <int> <chr>  <chr>
##  1   1871 NA    BS1    BNA      <NA>      3    31    NA    20    10 <NA>   <NA> 
##  2   1871 NA    CH1    CNA      <NA>      2    28    NA    19     9 <NA>   <NA> 
##  3   1871 NA    CL1    CFC      <NA>      8    29    NA    10    19 <NA>   <NA> 
##  4   1871 NA    FW1    KEK      <NA>      7    19    NA     7    12 <NA>   <NA> 
##  5   1871 NA    NY2    NNA      <NA>      5    33    NA    16    17 <NA>   <NA> 
##  6   1871 NA    PH1    PNA      <NA>      1    28    NA    21     7 <NA>   <NA> 
##  7   1871 NA    RC1    ROK      <NA>      9    25    NA     4    21 <NA>   <NA> 
##  8   1871 NA    TRO    TRO      <NA>      6    29    NA    13    15 <NA>   <NA> 
##  9   1871 NA    WS3    OLY      <NA>      4    32    NA    15    15 <NA>   <NA> 
## 10   1872 NA    BL1    BLC      <NA>      2    58    NA    35    19 <NA>   <NA> 
## # ... with 2,945 more rows, and 37 more variables: LgWin <chr>, WSWin <chr>,
## #   R <int>, AB <int>, H <int>, X2B <int>, X3B <int>, HR <int>, BB <int>,
## #   SO <int>, SB <int>, CS <int>, HBP <int>, SF <int>, RA <int>, ER <int>,
## #   ERA <dbl>, CG <int>, SHO <int>, SV <int>, IPouts <int>, HA <int>,
## #   HRA <int>, BBA <int>, SOA <int>, E <int>, DP <int>, FP <dbl>, name <chr>,
## #   park <chr>, attendance <int>, BPF <int>, PPF <int>, teamIDBR <chr>,
## #   teamIDlahman45 <chr>, teamIDretro <chr>, win_pct <dbl>
\end{verbatim}

\texttt{mutate()} allows you to create new variables in your data frame.

\hypertarget{example}{%
\subsubsection{Example}\label{example}}

We can combine many of these steps together. Consider the example below.

\begin{Shaded}
\begin{Highlighting}[]
\NormalTok{teams }\OperatorTok\StringTok{ }
\StringTok{  }\KeywordTok{select}\NormalTok{(yearID, name, W, G) }\OperatorTok\StringTok{ }
\StringTok{  }\KeywordTok{mutate}\NormalTok{(}\DataTypeTok{win_pct =}\NormalTok{ W }\OperatorTok{/}\StringTok{ }\NormalTok{G) }\OperatorTok\StringTok{ }
\StringTok{  }\KeywordTok{arrange}\NormalTok{(}\KeywordTok{desc}\NormalTok{(win_pct)) }\OperatorTok\StringTok{ }
\StringTok{  }\KeywordTok{filter}\NormalTok{(G }\OperatorTok{==}\StringTok{ }\DecValTok{162}\NormalTok{)}
\end{Highlighting}
\end{Shaded}

\begin{verbatim}
## # A tibble: 1,151 x 5
##    yearID name                    W     G win_pct
##     <int> <chr>               <int> <int>   <dbl>
##  1   2001 Seattle Mariners      116   162   0.716
##  2   1998 New York Yankees      114   162   0.704
##  3   1969 Baltimore Orioles     109   162   0.673
##  4   1970 Baltimore Orioles     108   162   0.667
##  5   1975 Cincinnati Reds       108   162   0.667
##  6   1986 New York Mets         108   162   0.667
##  7   2018 Boston Red Sox        108   162   0.667
##  8   2019 Houston Astros        107   162   0.660
##  9   1998 Atlanta Braves        106   162   0.654
## 10   2019 Los Angeles Dodgers   106   162   0.654
## # ... with 1,141 more rows
\end{verbatim}

\hypertarget{data-visualization}{%
\subsection{Data visualization}\label{data-visualization}}

The \texttt{ggplot2} package is a package in \texttt{tidyverse} that is
great for data wrangling data frame objects.

\hypertarget{basics}{%
\subsubsection{Basics}\label{basics}}

Every \texttt{ggplot} object starts with a call to function
\texttt{ggplot()}.

\begin{Shaded}
\begin{Highlighting}[]
\KeywordTok{ggplot}\NormalTok{()}
\end{Highlighting}
\end{Shaded}

\includegraphics{baseball_analysis_files/figure-latex/unnamed-chunk-14-1.pdf}

Add some data and variables you want to map to aesthetics.

\begin{Shaded}
\begin{Highlighting}[]
\KeywordTok{ggplot}\NormalTok{(}\DataTypeTok{data =}\NormalTok{ teams, }\KeywordTok{aes}\NormalTok{(}\DataTypeTok{x =}\NormalTok{ R, }\DataTypeTok{y =}\NormalTok{ W))}
\end{Highlighting}
\end{Shaded}

\includegraphics{baseball_analysis_files/figure-latex/unnamed-chunk-15-1.pdf}

Add a geometry. In this case we'll use \texttt{geom\_point()} to create
a scatterplot.

\begin{Shaded}
\begin{Highlighting}[]
\KeywordTok{ggplot}\NormalTok{(}\DataTypeTok{data =}\NormalTok{ teams, }\KeywordTok{aes}\NormalTok{(}\DataTypeTok{x =}\NormalTok{ R, }\DataTypeTok{y =}\NormalTok{ W)) }\OperatorTok{+}
\StringTok{  }\KeywordTok{geom_point}\NormalTok{()}
\end{Highlighting}
\end{Shaded}

\includegraphics{baseball_analysis_files/figure-latex/unnamed-chunk-16-1.pdf}

Finally, add some labels and a theme.

\begin{Shaded}
\begin{Highlighting}[]
\KeywordTok{ggplot}\NormalTok{(}\DataTypeTok{data =}\NormalTok{ teams, }\KeywordTok{aes}\NormalTok{(}\DataTypeTok{x =}\NormalTok{ R, }\DataTypeTok{y =}\NormalTok{ W)) }\OperatorTok{+}
\StringTok{  }\KeywordTok{geom_point}\NormalTok{() }\OperatorTok{+}
\StringTok{  }\KeywordTok{labs}\NormalTok{(}\DataTypeTok{x =} \StringTok{"Runs scored"}\NormalTok{, }\DataTypeTok{y =} \StringTok{"Wins"}\NormalTok{, }\DataTypeTok{title =} \StringTok{"More runs means more wins!"}\NormalTok{) }\OperatorTok{+}
\StringTok{  }\KeywordTok{theme_minimal}\NormalTok{()}
\end{Highlighting}
\end{Shaded}

\includegraphics{baseball_analysis_files/figure-latex/unnamed-chunk-17-1.pdf}

With \texttt{ggplot}, the plus operator, \texttt{+}, has a special
meaning. It links together the \texttt{ggplot2} functions to build your
visualization.

\hypertarget{pythagorean-wins}{%
\subsection{Pythagorean wins}\label{pythagorean-wins}}

Let's put together everything we know and create a scatterplot showing
the Pythagorean wins versus a team's win percentage for all teams since
2000.

\begin{Shaded}
\begin{Highlighting}[]
\NormalTok{teams }\OperatorTok\StringTok{ }
\StringTok{  }\KeywordTok{filter}\NormalTok{(yearID }\OperatorTok{>=}\StringTok{ }\DecValTok{2000}\NormalTok{) }\OperatorTok\StringTok{ }
\StringTok{  }\KeywordTok{ggplot}\NormalTok{(}\KeywordTok{aes}\NormalTok{(}\DataTypeTok{x =}\NormalTok{ R }\OperatorTok{^}\StringTok{ }\DecValTok{2} \OperatorTok{/}\StringTok{ }\NormalTok{(R }\OperatorTok{^}\StringTok{ }\DecValTok{2} \OperatorTok{+}\StringTok{ }\NormalTok{RA }\OperatorTok{^}\StringTok{ }\DecValTok{2}\NormalTok{), }\DataTypeTok{y =}\NormalTok{ W }\OperatorTok{/}\StringTok{ }\NormalTok{G)) }\OperatorTok{+}
\StringTok{  }\KeywordTok{geom_point}\NormalTok{(}\DataTypeTok{color =} \StringTok{"blue"}\NormalTok{) }\OperatorTok{+}
\StringTok{  }\KeywordTok{geom_abline}\NormalTok{(}\DataTypeTok{intercept =} \DecValTok{0}\NormalTok{, }\DataTypeTok{color =} \StringTok{"red"}\NormalTok{) }\OperatorTok{+}
\StringTok{  }\KeywordTok{labs}\NormalTok{(}\DataTypeTok{x =} \StringTok{"Pythagorean Wins"}\NormalTok{, }\DataTypeTok{y =} \StringTok{"Win percentage"}\NormalTok{,}
       \DataTypeTok{title =} \StringTok{"Teams above the line exceeded their Pythagorean wins"}\NormalTok{) }\OperatorTok{+}
\StringTok{  }\KeywordTok{theme_minimal}\NormalTok{(}\DataTypeTok{base_size =} \DecValTok{14}\NormalTok{)}
\end{Highlighting}
\end{Shaded}

\includegraphics{baseball_analysis_files/figure-latex/unnamed-chunk-18-1.pdf}

\end{document}
